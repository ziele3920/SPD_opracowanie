\section{Programowanie dynamiczne}
	\subsection{Charakterystyka metody}

		Schemat ten określa ogólne podejście polegające na przekształceniu zadania
		optymalizacji w wieloetapowy proces podejmowania decyzji, w którym
		stan na każdym etapie zależy od decyzji wybieranej ze zbioru decyzji dopuszczalnych.
		Stany poprzednich etapów
		zostają zapamiętane zatem eliminowana jest konieczność kilkukrotnego przeliczania tych 
		samych rozwiązań (porozwiązywań). Złożoność algorytmów dla tego podejścia może być 
		wielomianowa(max droga w grafie), pseudowielomianowa(problem załadunku) 
		jak i wykładnicza(TSP).
		Jest to oczywiście metoda generująca rozwiązanie optymalne problemu.
		
	\subsection{Przykładowe zastosowanie}
	
		Programowanie dynamiczne znajduje zastosowanie w popularnym problemie plecakowym 
		(patrz \ref{knapsack}).
		