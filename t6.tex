\section{Problem plecaka}

	\subsection{Sformułowanie}
		Danych jest $n$ przedmiotów, każdy o objętości(wadze) $w_i$ oraz cenie(wartości) $c_i$.
		Dany jest również plecak o pojemności $W$
		Należy zapakować do plecaka przedmioty tak, aby ich sumaryczna wartość była możliwie jak największa przy
		nie przekroczeniu objętości plecaka.
		
	\subsection{Własności}
		\begin{itemize}
			\item Problem jest NP-trudny.
			\item Występuje w postaci ciągłej jak i dyskretnej.
		\end{itemize}
	\subsection{Metoda rozwiązania}
		\begin{enumerate}
			\item Metody dokładne
			\begin{itemize}
				\item Przegląd zupełny - generuje wszystkie dopuszczalne rozwiązania i z nich wybiera optymalne $O(2^n)$.
				\item Programowanie dynamiczne - złożoność pseudowielomianowa. Dzieli zadanie na mniejsze - prostsze do 
				rozwiązania. Na początku przyjmuje, że plecak ma pojemność 1, następnie generuje optymalne rozwiązanie
				dla plecaka o takiej pojemności, zapamiętuje je i inkrementuje pojemność plecaka tym razem szukając 
				rozwiązania optymalnego korzysta z wcześniej znalezionego rozwiązania dla mniejszej objętości plecaka.
				Ten schemat jest powtarzany aż do osiągnięcia wymaganej pojemności plecaka wraz z rozwiązaniem optymalnym.
			\end{itemize}
			\item Metody przybliżone
			\begin{itemize}
				\item Algorytm zachłanny - polega na posortowaniu przedmiotów niemalejąco według stosunku ceny do wagi 
				$\frac{c_i}{w_i}$. Następnie iterując całą posortowaną kolekcję od pierwszego elementu
				 umieszcza kolejno w plecaku te przedmioty, które wraz z przedmiotami umieszczonymi wcześniej nie 
				 przekraczają pojemności plecaka aż do końca kolejki lub całkowitego zapełnienia plecaka. 
				 Złożoność algorytmu $O(n logn)$.
			\end{itemize}
		\end{enumerate}
