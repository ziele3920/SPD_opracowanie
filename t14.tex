\section{Modelowanie dodatkowych ograniczeń w problemach planowania}

	W praktyce obok problemów przepływowych odpowiadających modelom
	podstawowym pojawiają się problemy bardziej
	złożone. Najczęściej są one pochodnymi zagadnienia podstawowego otrzymanymi
	poprzez wprowadzenie dodatkowych ograniczeń o różnym charakterze.
	
	\subsection{Transport}
	Przemieszczenie zadania pomiędzy maszynami w wielu przypadkach praktycznych jest tak duża, że modeluje się je jako
	maszynę o nieograniczonej przepustowości. Co zwiększa jednak niepotrzebnie rozmiar modelu grafowego.
	Aby tego uniknąć wystarczy dodatkowo obciążyć krawędzie grafu łączące operacje wykonywane na różnych maszynach
	czasami transportów pomiędzy tymi maszynami.
	\subsection{Przezbrojenia}
	Przezbrojeniem nazywamy czas potrzebny na zmianę oprzyrządowania
	maszyny w związku z wykonywanym zadaniem. Najbardziej ogólny przypadek
	zakłada, że czas ten zależy od maszyny oraz pary kolejno realizowanych
	po sobie zadań.
	\subsection{Stanowska o nieograniczonej przepustowości}
	Maszyna ma przepustowość $k$ jeżeli w dowolnym momencie może wykonywać nie więcej niż $k$ operacji równocześnie.
	Zatem za maszynę o nieograniczonej przepustowości można uważać:
	\begin{itemize}
		\item urządzenie, które w sensie fizycznym pozwala obsługiwać jednocześnie
		dowolnie wiele zadań, np. piec grzewczy,
		\item maszynę o ograniczonej przepustowości, dla której czas trwania jest
		nieporównywalnie mały w stosunku do maszyn sąsiednich przez co nie
		obserwuje się występowania kolejki,
		\item zbiór urządzeń fizycznych, identycznych funkcjonalnie i o tak dużej
		liczności, że operacje realizowane na tych urządzeniach nie muszą być szeregowane,
		\item maszynę otrzymaną poprzez zagregowanie (połączenie) podzbioru kolejnych
		maszyn o nieograniczonej przepustowości,
		\item maszynę fikcyjna realizującą proces wymagający upływu czasu lecz nie
		angażujący urządzenia w sensie fizycznym (operacje starzenia, schnięcia,
		dojrzewania, chłodzenia, itp.).
	\end{itemize}
	Termin gotowości zadania możemy interpretować jako wykonanie fikcyjnej operacji zadania na maszynie o 
	nieograniczonej przepustowości. \newline
	Problemy zawierające maszyny o nieograniczonej przepustowości zachowują
	swoje własności związane ze ścieżką krytyczną w grafie oraz własności eliminacyjne dotyczące bloków zadań.
	\subsection{Terminy dostępności}
	Termin dostępności zadania możemy modelować jako wykonanie fikcyjnej operacji zadania na maszynie o 
	nieograniczonej przepustowości.
	
	\subsection{Bufory}
	Bufor jest miejscem do chwilowego składowania zadania przekazywanego
	pomiędzy maszynami i jest zwykle związany z maszyną (bufor do maszyny).
	W obszarze bufora tworzona jest kolejka zadań do obsługi. Pojemność
	bufora jest rozumiana jako maksymalna liczba zadań, które mogą być składowane
	równocześnie. Klasyczne problemy szeregowania przyjmują zwykle,
	że pojemność bufora jest dowolnie duża (nieograniczona). Zadania, które po
	zakończeniu wykonywania na maszynie nie mogą być przekazane do odpowiedniego
	bufora pozostają na niej powodując zablokowanie tak długo aż
	pojawi się wolne miejsce w buforze. Logiczne pojęcie bufora może modelować:
	\begin{itemize}
		\item fizyczne urządzenie procesu technologicznego,
		\item ograniczony fizycznie obszar składowania,
		\item fikcyjne wymaganie “wymuszajace” przepływ produkcji.
	\end{itemize}
	Generalnie, rozpatrywane są dwie kategorie problemów: (1) z buforem o
	zerowej pojemności (ograniczenie no store, strategia NS), (2) z buforami o
	skończonej, ograniczonej pojemności, różnej dla różnych maszyn (strategia
	LS).
	
	\subsection{Czas cyklu}
	Jest to ograniczenie, które zakłada cykliczność produkcji zatem maksymalny czas wykonania wszystkich zadań
	jest ograniczony czasem cyklu. Gdy nie ma możliwości realizacji wszystkich zadań w jednym cyklu należy rozdzielić 
	je na kilka cykli.