\section{Modele harmonogramowania z dyskretnym czasem. Problem z jednym stanowiskiem obsługi}
Harmonogramowanie produkcji: rozłożenie w czasie przydziału zasobów do zleceń produkcji, podzielenie zadań na partię produkcyjne, określenie terminów rozpoczęcia i zakończenia realizacji partii na poszczególnych maszynach. Kompromis pomiędzy: kosztami niedotrzymania terminów, zaspokojeniem zapotrzebowania, kosztami utrzymania zapasów i zmian produkcji. 
Koszta zmiany produkcji wiążą się z przestojem maszyn, gdyż zachodzi potrzeba ich przezbrojenia.
W przypadku jednego stanowiska obsługi problem jest o tyle trudny, że nie można wykonywać dwóch typów zadań
bez przestoju. Istnieją dwa rodzaje podejść:
\begin{enumerate}
	\item Długi cykl i mała liczba przezbrojeń - w tym podejściu cykl trwa długo zatem produkowana jest znaczna ilość
	półproduktów co wymaga dużych zasobów magazynowych, jednakże w dłuższym okresie czasu minimalizuje liczbę przestojów 
	związanych z przezbrajaniem stanowiska. Produkty są wypuszczane rzadko w dużych ilościach.
	\item Krótki cykl i duża liczba przezbrojeń - cykl jest możliwie krótki dzięki czemu nie jest konieczna duża przestrzeń
	magazynowa i następuje cykliczne wypuszczanie małych serii produktów. Jednakże w dłuższym okresie czasu liczba 
	przestojów jest znaczna.
\end{enumerate}