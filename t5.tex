\section{Problem komiwojazera. Sformulowanie, wlasnosci i metoda rozwiazywania}

\subsection{Sformulowanie}
Dane jest n miast, ktore komiwojazer musi odwiedzic, oraz odleglosci miedzy kazda para miast. Celem jest znalezienie najkrotszej drogi laczacej wszystkie miasta zaczynajacej i konczocej sie w okreslonym punkcie. Sprowadza sie to do budowy grafu gdzie wierzcholki to miasta a wagi krawedzi to odleglosci miedzy nimi.
\subsection{Wlasnosci}
\begin{itemize}
\item Z uwagi na to ze powstaly graf jest grafem pelnym to na pewno posiada przynajmniej jeden minimalny cykl Hamiltona ( problem zawsze ma rozwiazanie).
\item Zagadnianie nalezy do problemow NP-trudnych - duza zlozonosc obliczeniowa wraz ze wzrostem liczby miast (nie wiadomo czy mozna rozwiazac w czasie wielomianowym).
\end{itemize}


\subsection{Metody}
Do rozwiazywania tego rpoblemu stosuje sie metody przyblizone.
Za pomoca metod programowania dynamicznego istnieje algorytm 'Held-Karp algorithm' (Helda-Karpia :D ?) ktory umozliwia rozwiazanie problemu w czasie$ O(n^22^n)$
\\ Ale algorytmy dokladne wolno dzialaja i raczej stosuje sie przyblizone. Algorytm mrowkowy, Lin-Kernighan, NN ( nearest neighbour)

Algorytm 2-optymalny - W podejściu tym bazujemy na obserwacji, iż krzyżujące się połączenia między miastami są zawsze gorsze niż takie, które się nie krzyżują.
W algorytmie tym zatem sprawdza się wszystkie możliwe pary krawędzi i jeśli którakolwiek
zawiera krawędzie krzyżujące się, następuje takie przestawienie czterech miast na trasie, by
krzyżujące się krawędzie zostały zastąpione przez takie, które się nie krzyżują. Jednakże, brak
krzyżujących się krawędzi wcale nie gwarantuje optymalności rozwiązania i cały proces
przeważnie kończy się w minimum lokalnym. Aby „uciec” z tego minimum lokalnego
wprowadzić można losowe zaburzenia do aktualnie najlepszej trasy 

