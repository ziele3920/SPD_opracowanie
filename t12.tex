\section{Balansowanie linii montażowej.
Przykładowy problem i metoda rozwiązywania.
Związek z 
szeregowaniem.}

\subsection{Sformułowanie}
Balansowanie linii montażowej (assembly line balancing) polega na równomiernym rozłożeniu operacji między stacje robocze linii tak, aby czas przestoju poszczególnych maszyn był minimalny. Zakłada się znajomość czasu wykonywania operacji na maszynach, relacje kolejnościowe pomiędzy operacjami oraz wielkość cyklu produkcyjnego lub liczby maszyn. Aby proces balansowania był zakończony należy przydzielić każdą operację raz i tylko do jednej stacji roboczej.
		
\subsection{Własności}
Wyróżnia się dwa typy zagadnienia ALBP (assembly line balancing problem):
\begin{enumerate}
\item ALBP I - celem jest uzyskanie minimalnej liczby stacji roboczych przy założeniu stałego i znanego czasu cyklu
\item ALBP II - celem jest określenie minimalnej wartości cyklu przy stałej i znanej liczbie stacji roboczych
\end{enumerate}
Docelowo rozwiązanie obu problemów dąży do zwiększenia efektywności linii produkcyjnej - dla typu pierwszego jest to minimalizacja kosztów poprzez redukcję całkowitej ilości godzin pracy (np. zmniejszenie ilości pracowników). Dla typu drugiego skrócenie cyklu zwiększa przepustowość linii.
	
		
\subsection{Metoda rozwiązania}
