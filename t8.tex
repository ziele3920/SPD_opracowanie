\section{Optymalizacja pracy linii wytwórczej (problem przepływowy).
Sformułowanie, właściwości
i metoda rozwiązywania.}

	\subsection{Sformułowanie}
		Zbiór zadań $J=(1,2,...,n)$ jest przeznaczony do wykonania w podanej kolejności na $M=(1,2,...,m)$ maszynach o ograniczonej jednostkowej przepustowości. Każde zadanie $j\in J$ składa się z ciągu operacji $(O_1j, ..., O_mj)$. Operacja $O_ij$ odpowiada nieprzerywalnemu wykonywaniu zadania $j$ na maszynie $i$ w czasie $p_ij$. Rozwiązaniem jest harmonogram pracy maszyn reprezentowany przez macierze terminów rozpoczęcia oraz zakończenia zadań spełniające powyższe ograniczenia. W praktyce rozwiącanie jest całkowicie określone przez jedną z macierzy, gdyż aby otrzymać drugą wystarczy dodać/odjąć czasy wykonywania zadań $p_ij$. 

		
	\subsection{Własności}
	Problemy przepływowe są NP-trudne (wyjąwszy niektóre szczególne przypdaki).
	Dzielą się na dwa rodzaje:
\begin{enumerate}
\item Ogólne - gdy kolejność wykonywania zadań na maszynie może być różna dla każdej maszyny
\item Permutacyjne - gdy wszystkie permutacje są takie same (taka sama kolejność wykonywania zadań na wszystkich maszynach)
\end{enumerate}
	Problem permutacyjny jest częściej analizowany, głównie z powodu znacznie mniejszej ilości rozwiązań ($n!$, podczas gdy $n!^m$ dla problemu ogólnego. Często błąd pomiędzy rozwiązaniami jest optymalnymi obu typów problemów jest nieznaczny, czasem nawet rozwiązanie optymalne problemu ogólnego leży w klasie rozwiązań permutacyjnych. Rozwiązania problemów permutacyjnych mogą być wykorzystywane jako rozwiązania początkowe w algorytmach przybliżonych dla problemów ogólnych.
	
		
	\subsection{Metoda rozwiązania}
	Algorytm NEH - oparty na technice wcięć, do tej pory najlepszy wśród konstrukcyjnych algorytmów przybliżonych dla problemu permutacyjnego. Składa się z n-krokowej fazy zasadniczej poprzedzonej fazą wstępną. Zadania są sortowane nierosnąco po sumie czasów wykonań na maszynach. W fazie zasadniczej, w $j$-tym kroku, do istniejącej aktualnie permutacji, dokładane jest $j$-te zadanie z kolejki zadań wcześniej posortowanych. Jest ono wstawiane we wszystkie możliwe miejsca w akutalnej permutacji, dostarczając $j$ nowych permutacji. Permutacja o najmniejszej wartości funkcji celu przyjmowana jest jako najlepsza w tym kroku i uznawana za aktualną. 
	Analogią działania jest pakowanie torby na wyjazd - najpierw pakujemy jeden lub kilka największych elemetów, po czym kolejne coraz mniejsze elementy metodą "dopychamy" metodą prób.