\section{Przykładowe zadania optymalizacji. Klasyfikacja podejść i metod}
	\subsection{Przykładowe zadania optymalizacji}
	\begin{itemize}
		\item Problem plecakowy - zadanie to polega na zapakowaniu do plecaka przedmiotów
		 tak, aby osiągnąć maksymalną sumaryczną wartość przedmiotów zapakowanych
		 przy ograniczonej pojemności plecaka.
		 \item Problem komiwojażera (TSP) - polega na znalezieniu minimalnego cyklu Hamiltona
		 w pełnym grafie ważonym. Odwzorowaniem tego problemu w rzeczywistości jest 
		 rozwiązanie problemu podróżnego handlarza, który chce odwiedzić wszystkie
		 zaplanowane miasta minimalizując jednoczenie drogę lub czas lub koszt odbycia tej
		 podróży.
		 \item Optymalizacja procesu wytwarzania - uszeregowanie zadań w taki sposób aby
		 zostało osiągnięte zadane kryterium np. minimalizacja czasu wykonania wszystkich
		 zadań.
	\end{itemize}
	\subsection{Klasyfikacja podejść i metod}
	\begin{enumerate}
		\item Metody dokładne
			\begin{itemize}
				\item Schemat podziału i ograniczeń (B\&B)
				\item Schemat programowania dynamicznego
				\item Programowanie liniowe całkowitoliczbowe
				\item Programowanie liniowe binarne
				\item Metody subgradientowe
			\end{itemize}
	\end{enumerate}