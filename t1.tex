\section{Przykładowe zadania optymalizacji. Klasyfikacja podejść i metod}
	\subsection{Przykładowe zadania optymalizacji}
	\begin{itemize}
		\item Problem plecakowy - zadanie to polega na zapakowaniu do plecaka przedmiotów
		 tak, aby osiągnąć maksymalną sumaryczną wartość przedmiotów zapakowanych
		 przy ograniczonej pojemności plecaka.
		 \item Problem komiwojażera (TSP) - polega na znalezieniu minimalnego cyklu Hamiltona
		 w pełnym grafie ważonym. Odwzorowaniem tego problemu w rzeczywistości jest 
		 rozwiązanie problemu podróżnego handlarza, który chce odwiedzić wszystkie
		 zaplanowane miasta minimalizując jednoczenie drogę lub czas lub koszt odbycia tej
		 podróży.
		 \item Optymalizacja procesu wytwarzania - uszeregowanie zadań w taki sposób aby
		 zostało osiągnięte zadane kryterium np. minimalizacja czasu wykonania wszystkich
		 zadań.
	\end{itemize}
	\subsection{Klasyfikacja podejść i metod}
	\begin{enumerate}
		\item Metody dokładne
			\begin{itemize}
				\item Schemat podziału i ograniczeń (B\&B) - ogólne podejście oparte na dekompozycji
				(podziału na mniejsze problemy, redukcja ograniczeń) i
				"inteligentnym" przeszukiwaniu zbioru rozwiązań dopuszczalnych problemu optymalizacyjnego.
				Znajduje zastosowanie w problemach silnie NP-trudnych. Dostarcza algorytmów o wykładniczej
				złożoności obliczeniowej. Może być stosowany dla dowolnego problemu dyskretnego (liniowego i
				nieliniowego).
				\item Schemat programowania dynamicznego (PD) - podejście polegające na przekształceniu zadania
				optymalizacji w wieloetapowy proces podejmowania decyzji, w którym stan na każdym etapie
				zależy od decyzji wybieranej ze zbioru decyzji dopuszczalnych. Stany poprzednich etapów
				zostają zapamiętane zatem eliminowana jest konieczność kilkukrotnego przeliczania tych 
				samych rozwiązań (porozwiązywań). Złożoność algorytmów dla tego podejścia może być 
				wielomianowa(max droga w grafie), pseudowielomianowa(problem załadunku) 
				jak i wykładnicza(TSP).
				\item Programowanie liniowe całkowitoliczbowe (PLC) - podejście w którym zarówno funkcja celu 
				jak i zestaw ograniczeń składają się z funkcji liniowych z żądaniem aby wszystkie 
				parametry były wyrażone liczbami całkowitymi.
				\item Programowanie liniowe binarne (PLB) - tak jak PLC, w którym parametry przyjmują 
				wartości binarne (0,1).
				\item Metody subgradientowe - mogą byś stosowane dla przypadków, gdzie funkcja celu jest
				funkcją ciągłą i subróżniczkowalną, posiadającą skończoną wartość ekstremum, a zbiór 
				rozwiązań jest niepusty, domknięty i wypukły. Metody te s ą kosztowne obliczeniowo, a ich
				szybkość zbiegania do rozwiązania optymalnego silnie zależy od przykładu problemu.
			\end{itemize}
		\item Metody przybliżone - są to metody, które nie znajdują rozwiązania optymalnego, lecz
		rozwiązanie bliskie optymalnemu. Są stosowane tam, gdzie ważniejsze jest szybkie otrzymanie 
		rozwiązania.
			\begin{itemize}
				\item Konstrukcyjne - szybkie, łatwe w implementacji lecz rozwiązanie dość znacznie
				odbiega od rozwiązania optymalnego.
				\item Poprawiające - wolniejsze, wymagają podania początkowego rozwiązania rozwiązania,
				które poprawiają w kolejnych krokach. Dostarczają rozwiązań o bardzo dobrej i doskonałej
				jakości. Umożliwiają kształtowanie kompromisu pomiędzy jakością a czasem obliczeń.
			\end{itemize}
	\end{enumerate}