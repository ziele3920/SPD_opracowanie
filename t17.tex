\section{Programowanie liniowe całkowitoliczbowe. Sformułowanie, metody
rozwiązywania, przykład zastosowania.}
\subsection{Sformułowanie}
Programowanie liniowe, w którym na zmienne decyzyjne (niektóre lub wszystkie) nałożono dodatkowe warunki, że muszą przyjmować wartości całkowite dodatnie, ponieważ rozwiązania z wartościami ułamkowymi nie miałyby sensu rzeczywistego (np. określenia ⅔ osoby lub ¾ samochodu).

W zagadnieniach programowania liniowego z reguły nie jest możliwe stosowanie zaokrągleń rozwiązań z wartościami ułamkowymi do najbliższych liczb całkowitych, gdyż wynik takiego postępowania może być daleki od rozwiązania optymalnego; może też nie spełniać warunków ograniczających. Przy programowaniu całkowitoliczbowym zachodzi więc potrzeba stosowania metod uwzględniających te warunki.

Problemy programowania całkowitoliczbowego należą do klasy NP-zupełnej.


\subsection{Metody rozwiązywania}
Wyróżnia się trzy podejścia do rozwiązywania zagadnień programowania całkowitoliczbowego
\begin{itemize}
\item metody przeglądu pośredniego (niebezpośredniego), m.in. metody podziału i ograniczeń,
Metoda podziału i ograniczeń jest oparta na podejściu “dziel i zwyciężaj”.
Kluczowe fakty:\newline
PCL= LP + ograniczenia całkowitoliczbowości\newline
Fakt 1. Wartość optymalna funkcji celu LP jest górnym ograniczeniem (maksymalizacja
funkcji celu) optymalnej wartości funkcji celu PCL.\newline
Fakt 2. Wartość funkcji celu PCL dla dowolnego rozwiązania całkowitoliczbowego
jest dolnym ograniczenie\newline
\item metody płaszczyzn odcinających,
\item metody oparte na dekompozycji (podziale).
\end{itemize}


\subsection{Przykład zastosowania}
\begin{itemize}
\item Wyznaczenie optymalnego planu produkcji maksymalizującego łączny zysk
\item Wyznaczenie planu rozkroju desek minimalizując łączny odpad
\item Problem plecakowy
\end{itemize}
