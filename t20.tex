\section{Przyblizone metody rozwiazywania zadan optymalizacji. Miary i metody oceny}

\subsection{Tak ogolnie}}
Przyblizone metody rozwiazywania zadan optymalizacji wyznaczaja takie rozwiazanie ktore jest blisko rozwiazania optymalnego. Z uwagi na to ze poruszane problemy sa zazwyczaj NP-trudne to nie dziwi ze takich metod jest wiecej.
Miary oceny "dobroci" metody:
\begin{enumerate}
\item  zlozonosc obliczeniowa algorytmu
\item dokladnosc przyblizenia
\item gwarancja zbieznosci do rozwiazania optymalnego
\item szybkosc zbieznosci do rozwiazania optymalnego
\end{enumerate}

\\ Blad przyblizenia algorytmu mozna liczyc na wiele roznych sposobow np tak:
$BLAD = r. optymalne - r.otrzymane $\\
$BLAD = r. optymalne /r.otrzymane $\\
$BLAD = r. \frac{(r.optymalne - r.otrzymane)}{r.optymalne} $\\
$BLAD = r. \frac{(r.optymalne - r.otrzymane)}{r.otrzymane} $\\


Blad moze byc badany experymentalnie i teoretycznie - experyment najczesciej bo latwo, ale jest to subiektywne bo zalezy od probki przykladow. Dopiero wyniki analizy teoretycznej ( najgorsze przypadki etc.) w polaczeniu z wynikami analizy experymentalnej oraz zlozonosci obliczeniowej stanowia kompletna charakterystyke algorytmu.
\\

\subsection{Analiza experymentalna}
Najpopularniejsza, niedokladna. Ocena a posteriori zachowania sie algorytmu (blad przyblizenia, czas pracy) w oparciu o wynik przebiegu na nieprecyzyjnej acz reprezentatywnej probce. Otrzymane wyniki nie zawsze dostatecznie dobrze odzwieciedlaja wlasnosci numeryczne algorytmu, gdyz nie zawsze do konca wiadomo co oznacza pojecie probki reprezentatywnej
\subsection{Analiza najgorszegoprzypadku}
Analiza najgorszego przypadku ocenia a priori zachowanie sie wybranego bledu na calej populacji przykladow. Wyznacza sie wspolczynnik najgorszego przypadku i asymptotyczny wspolczynnik najgorszego przypadku (trudne wzorki).  Taka analiza dostarcza skrajnie pesymistycznych wnioskow, nie jest powiedziane ze w praktyce taka sytuacja kiedokolwiek wystapi.

\subsection{Analiza probabilistyczna}
Analiza aprioryczna, zaklada ze kazdy przyklad zostal otrzymany jako realizacja pewnej niezaleznej zmiennej losowej o znanym rozkladzie ( najczesciej rown. ). W takim podejsciu blad przyblizenia rowniez jest zmienna losowa. Przy tych badaniach wnioskuje sie o jego rozkladach momentach a co najwazniejsze zbieznosci do jakiejs wartosci stalej wraz ze wzrostem liczby probek oraz o szybkosci tej zbieznosci. Bardzo skomplikowana analiza, dostarcza niezlych wynikow jednak malo algorytmow ma takie opracowanie bo trudno sie to robi.
