\section{Strategie wytwarzania. Systemy sterowania}
\subsection{Systemy sterowania}
System sterowania ma zapewnic uruchamianie, nadzorowanie i zapewnienie realizacji zadan produkcyjnych. W zaleznosci od wielkosci produkcji, jej charakteru linii produkcyjnej i stoppnia automatyzacji parku maszynowego stosowane sa rozne strategie wytwarzania.

\subsection{Strategie wytwarzania}
\subsubsection{PUSH}
Zadania wytworcze (zamowienia na produkt koncowy) sa tlumaczone na zadania materialow i polproduktow a nastepnie przepychane przez system sterowania produkcj? wed?ug ustalonego harmonogramu. W razie potrzeby harmonogram jest na bierzaco korygowany i odpowiednie sterowania sa przekazywane do systemu wytwarzania. Sterowanie tego typu nosi nazwe nadaznego

Systemy sterowania dla tej strategii to MRP i ERP.
\begin{itemize}
\item MRP - Material Requirements planning -  Umozliwia kontrole rodzajow ilosci i terminow produkcji a takze sterowanie zapasami i ich uzupelnianiem
\item ERP - enterprise resource planning - system wspomagajacy nadzor nad calym procesem produkcji poczawszy od zaopatrzenia w materialy a skonczywszy na dostawie do odbiorcy.
\end{itemize}

Strategie polecana dla produkcji jednostkowej i krotkoseryjnej.
\subsubsection{SQUEEZE}
Strategia zak?ada ?e wydajnosc systemu wytworczego jest ograniczona przepustowoscia waskeigo gardla systemu. Gard?o to zestaw stanowisk wytworczych przez ktore produkcja sie przeciska powodujacspietrzanie i kolejki zadan.

Jeno jeden system OPT : 
umozliwia zoptymalizowanie przeyplywu produkcji koncentrujac sie na waskim gardle upatrujac w nim element determinujacy dzialanie calego systemu produkcyjnego.

Strategia polecana dla produkcji krotko i srednio seryjnej.
\subsubsection{PULL ( JIT)}
Strategia zaklada za podstawe produkcji zgloszona wielkosc zapotrzebowania na okreslony produkt koncowy ktory powoduje ssanie na wyjsciu systemu. Przeklada sie to na ssanie materialow i polproduktow. Brak ssania oznacza bezczynnosc systemu i stanowisk wytwarzania , zapobiega zbednemu wytwarzaniu redukuje zapasy. 

KANBAN TOYOTA zamowienia gotowe na czas, karteczki etc. to chyba wiemy

\subsubsection{Inne strategie}
\begin{enumerate}
\item CAW - steruje zleceniami w celu zapewnienia stalego sredniego obciazenia stanowisk. Dobra jak stale terminy dostaw , stabilne dostawy materialow, niezmienna zdolnosc produkcyjna
\item CRS - ciagle uzupelnianie stawnow materialowych. Dobra dla produkcji seryjnej i powtarzalnej przy stalym zapotrzebowaniu

\end{enumerate}