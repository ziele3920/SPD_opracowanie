\section{Optymalizacja magazynowania. Przykładowy problem i metoda
rozwiązywania}
\subsection{Przykładowy problem}

Optymalizacja Procesów Magazynowych dotyczy czynności związanych z przyjmowaniem, składowaniem, kompletacją oraz wysyłką towarów w magazynach dowolnego typu (w tym wysokiego składowania). Pozwala na znalezienie oraz wyeliminowanie istniejących tzw. „wąskich gardeł”, określenie wymaganej wielkości poszczególnych stref obsługujących procesy magazynowe (w tym m.in. wielkość buforów na wejściu i na wyjściu z magazynu, centrum logistycznego, terminalu przeładunkowego czy innych obiektów infrastruktury logistycznej), określenie wymagań co do technologii zastosowanej w przypadku każdego z procesów magazynowych i wymaganej infrastruktury magazynowej. Może być zastosowane także w trakcie projektowania nowego magazynu w celu empirycznego określenia możliwych granic przepustowości magazynu jeszcze przed akceptacją jego projektu. Pozwala to na określenie niezbędnych do obsługi zasobów technicznych, w tym także do określenia optymalnego stopnia mechanizacji i automatyzacji procesów magazynowych.
Optymalizacja Procesów Magazynowych pozwala na:
\begin{itemize}
\item zwiększenie elastyczności procesów magazynowych
\item przyspieszenie przepływów towarów
\item eliminację lub redukcję problemu „wąskich gardeł”
\item skrócenie czasu przejścia towaru przez system logistyczny
\item obniżenie kosztów obsługi procesów magazynowych
\item optymalizację stref wykorzystywanych w przypadku realizacji poszczególnych procesów magazynowych
\item podniesienie poziomu obsługi
\end{itemize}

\subsection{Metoda rozwiązania}
Optymalizacja magazynowania wiąże się z identyfikacją kosztów utrzymania magazynu i ich częściowym obniżeniem bez uszczerbku na jakości wykorzystywanych materiałów eksploatacyjnych czy zagrożenia dla bezpieczeństwa pracy.
\subsubsection{Optymalizacja czasu pracy}
Na miejscu pracy powinna się znajdować optymalna liczba pracowników – niezbędna do realizacji wszystkich zadań.
\subsubsection{Obniżenie rachunków za energię elektryczną}
Koszty prądu w magazynach są znaczące. Można je jednak obniżyć, negocjując indywidualną stawkę z dostawcą energii. Warto rozmawiać nie tylko z dominującymi podmiotami, lecz także z mniejszymi dostawcami energii elektrycznej. Często ci mniejsi mają konkurencyjne oferty gwarantujące np. stałą cenę prądu na kilka lat. Gwarancja taka daje pewność, że koszty energii nie wzrosną, nawet gdy dojdzie do zmiany przepisów w prawie energetycznym.
\subsubsection{Zarządzania paletami}
Używanie przez magazyn drewnianych palet jest dla każdego przedsiębiorstwa kosztowne. Nie chodzi wyłącznie o ich zakup czy magazynowanie, lecz także o koszty transportu. Drewniane, ciężkie i wysokie palety można zastąpić tekturowymi, które przez to, że mogą być niższe, pozwalają załadować do samochodu ciężarowego więcej towaru. Dodatkowo są one lżejsze, co nie jest bez znaczenia dla kosztów paliwa. Koszty transportu spadają również z jeszcze jednego powodu. Papierowe palety są biodegradowalne i można je zostawić u klienta wraz z towarem. Zaleta tego rozwiązania jest taka, że zamiast wracać samochodem załadowanym wyłącznie paletami bez towaru, przed wysłaniem w drogę powrotną można go ponownie zatowarować.
\subsubsection{System identyfikacji pozycji magazynowej}
Im większy magazyn, tym trudniejsze jest znalezienie w nim np. małych partii produktów. Ważne jest, by pracownicy magazynu mieli możliwość natychmiastowego namierzenia poszczególnych produktów, a także identyfikacji poszczególnych pozycji. System automatycznej identyfikacji pozycji ułatwia pracę, jednak jest również wymagający – należy pamiętać o dokładnym wprowadzeniu produktu do systemu przy jego przyjęciu oraz o informowaniu o każdym jego późniejszym ruchu. Za pomocą tego systemu można w pełni automatycznie optymalizować alokację obiektów na regałach, co powoduje znaczące oszczędności w wykorzystaniu powierzchni magazynowej, a także czasu.
\subsubsection{Nowoczesny system IT}
Każdy nieplanowany przestój linii produkcyjnej generuje ogromne straty dla przedsiębiorstwa. Kosztochłonne są również zbyt wysokie stany zapasów (znaczące koszty magazynowania) oraz braki w magazynie. Dbanie o stany magazynowe jest ważne, gdyż umowy z kontrahentami obejmują wysokie kary za nieterminowe dostawy towarów. Można jednak nad tym zapanować, zlecając specjalistycznej firmie IT stworzenie dedykowanego systemu informatycznego, który będzie informował szefów działów (np. magazynu, zaopatrzenia) o brakach lub o stanach magazynowych. System taki może w znaczący sposób usprawnić pracę całego przedsiębiorstwa.