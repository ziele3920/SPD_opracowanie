\section{Optymalizacja procesu wytwarzania. Szeregowanie.}
	\subsection{Optymalizacja procesu wytwarzania}
	
	Proces wytwarzania jest rozbudowanym zagadnieniem. W celu jego optymalizacji należy zwrócić uwagę na:
	\begin{itemize}
		\item synchronizację terminów dostaw z zapotrzebowaniami,
		\item odpowiednim przydzieleniu w czasie zasobów do wykonywanych zadań,
		\item podział zadań na partie produkcyjne,
		\item uszeregowanie zadań (określenie ich terminów wykonywania na poszczególnych maszynach)	
	\end{itemize}		
	
	\subsection{Szeregowanie}
	Szeregowanie zadań w procesie wytwarzania jest kluczowym elementem optymalizacji tego procesu.
	Na podstawie problemu praktycznego tworzony jest opis przy użyciu pojęć z teorii szeregowania, który
	prowadzi do matematycznego modelu procesu. Symboliczny opis problemu szeregowania:
	
	\begin{equation}
		\alpha|\beta|\gamma
	\end{equation}
	
	$\alpha$ - typ zagadnienia, \newline
	$\beta$ - dodatkowe ograniczenia, \newline
	$\gamma$ - postać funkcji celu. \newline
	
	W typie zagadnienia zawarte są informacje o ilości maszyn, sposobie przejścia zadań przez system 
	(typ zagadnienia: przepływowy, gniazdowy, równoległy) oraz o trybie realizacji poszczególnych operacji zadania. \newline
	Pośród przykładowych dodatkowych ograniczeń można wymienić takie jak: prec - narzucony, częściowy porządek 
	wykonywania zadań, pmtn - dopuszczenie możliwości przerwania wykonywania zadania, setup - wstępują czasy 
	przezbrojenia maszyn pomiędzy wykonywaniem zadań i inne.  \newline
	Przykładowa funkcja celu może być w postaci minimalizacji czasu wykonania wszystkich zadań. \newline
	
	Problem szeregowania zazwyczaj jest problemem NP-trudym. Istnieje wiele algorytmów dokładnych jak i przybliżonych
	rozwiązujących problemy tego typu.
	