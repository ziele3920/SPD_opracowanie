\section{Optymalizacja systemu opartego na przepływie zadań}
	\subsection{Sformułowanie}
		Dane są:
		\begin{itemize}
			\item zbiór zadań $J={1, ..., n}$,
			\item zbiór maszyn $M={1, ..., m}$,
			\item zbiór operacji $O={1, ..., o}$
		\end{itemize}				
		Zbiór operacji jest dekomponowany na podzbiory odpowiadające zadaniom. Zatem zadanie $j$ składa się 
		z sekwencji $o_j$ operacji, które powinny zostać wykonane w zadanej kolejności (zgodnie z kolejnością w podzbiorze 
		$o_j$). Ponadto każda operacja musi zostać wykonana na przypisanej do niej maszynie, a maszyna może wykonywać
		tylko jedną operację w danej chwili czasu. \newline
		Na rozwiązanie dopuszczalne składa się wektor czasów rozpoczęcia wszystkich operacji. Najczęstszą formą funkcji celu
		jest minimalizacja $C_{max}$ - terminu zakończenia wszystkich zadań.
		
	\subsection{Właściwości}
		\begin{itemize}
			\item Problem można modelować za pomocą acyklicznego grafu $G(W)$, ($W$ - kompletna reprezentacja dopuszczalna).
			\item Problem jest NP-trudny.
		\end{itemize}			
	
	\subsection{Metoda rozwiązania}
		Jedną z metod rozwiązania problemu gniazdowego jest skorzystanie z algorytmu aproksymacyjnego.
		Nie generuje on rozwiązania optymalnego, jednakże jest bardzo wydajny obliczeniowo.
		Podstawowy algorytm aproksymacyjny składa się  z trzech kroków:
		\begin{enumerate}
			\item Wygeneruj rozwiązanie $S$ spełniające tylko wymagania porządku technologicznego tzn. zachowującego
			odpowiednią kolejność wykonywania operacji dla każdego zadania. Rozwiązanie takie jest niedopuszczalne, gdyż
			więcej niż jedno zadanie może zostać przydzielone do maszyny w tym samym momencie czas.
			W tym wypadku $C_{max}$ (dolne ograniczenie problemu gniazdowego) przyjmuje wartość $LB_J$ (suma czasów 
			wykonania najdłuższego zadania).
			\item Zaburz terminy rozpoczęcia operacji każdego zadania $i$ o wielkość $\delta_i$. Gdzie $\delta_i$
			jest całkowitą liczbą losową z rozkładu równomiernego na przedziale $[0, LB_M]$, gdzie $LB_M$ to suma czasów
			operacji na najbardziej "zajętej" (pracującej najdłużej) maszynie.
			\item “Rozciągnij” i “spłaszcz” otrzymane uszeregowanie tak, by w każdym
			momencie czasu na każdej maszynie było wykonywane nie więcej niż
			jedno zadanie.
		\end{enumerate}
		
		Inne metody służące do rozwiązania problemu gniazdowego:
		\begin{itemize}
			\item Schemat $B\&B$
			\item Algorytmy priorytetowe
			\item Przeszukiwania lokalne
			\item Metoda przesuwnego wąskiego gardła
			\item Symulowane wyżarzanie
			\item Poszukiwanie z zakazami
			\item Spełnianie ograniczeń
			\item Poszukiwanie ewolucyjne
			\item Podejście dualne
			\item Sieci neuronowe
		\end{itemize}